\chapter{Conclusion}

\section{Difficultés rencontrées}

Lors de ce projet, certaines difficultés ont su freiner notre progression et nous pousser à la réflexion, notamment la découverte et le développement d'un jeu sous forme de \textbf{client/serveur} qui au départ semblait complexe. L'implémentation du monde a également posé problème au début du projet, car les pistes de réflexions n'aboutissaient pas à une solution concluante. Cependant, les matrices GSL ont été le plus gros frein. Ces dernières étant nouvelles pour le groupe, leur utilisation n'était pas chose aisée bien qu'une documentation soit à notre disposition. L'extraction et la configuration de valeurs dans ces matrices nous a posé problème pendant un certain temps.    

\section{Bilan du projet}
Au final, le rendu final est un jeu  des amazones avec un plateau classique pouvant aller jusqu'à plus de $150 \times 150$. Ce plateau est représenté à l'aide d'un graphe de liaison qui n'est pas modifié au cours de la partie. Sur ce plateau, deux joueurs peuvent s'affronter avec chacun une stratégie différente et une représentation du monde qui leur est propre. Une partie serveur est également présente afin de lier l'ensemble et de vérifier le bon déroulement de la partie. Ainsi, à la fin de chaque match, un vainqueur est désigné et les joueurs créés en local, après l'exportation de leur bibliothèque, peuvent aller affronter d'autres joueurs appartenant aux autres équipes de projet. 
\\
Cependant, le projet est loin d'être complet. Ci-dessous se trouvent quelques idées de points à améliorer ou à implémenter :
\medbreak
\begin{itemize}
    \item \textbf{Types de monde}: différents types de monde nous ont été proposés et nous n'avons eu le temps d'en réaliser qu'un seul qui fonctionne correctement,
    \item \textbf{Stratégie}: ce point peut être amélioré de plusieurs façons, la meilleure étant de développer une forme d'IA qui choisirait à chaque fois le meilleur coup possible pour réussir à vaincre l'adversaire,
    \item \textbf{Formation de départ}: dans l'optique, cette fois-ci, de légèrement modifier le jeu, de nouvelles formations de départ pour les reines peuvent être un changement à effectuer.
\end{itemize}

\medbreak

Pour conclure, ce projet aura permis d'adopter une nouvelle approche sur le fonctionnement et le développement d'un jeu de plateau qui se voudrait multijoueurs. La participation des étudiants de deuxième année a permis de mieux comprendre le fonctionnement de l'IA et les différentes méthodes utilisées pour maximiser les chances de jouer le meilleur coup. Cette expérience a également stimulé la réflexion sur le sujet.
