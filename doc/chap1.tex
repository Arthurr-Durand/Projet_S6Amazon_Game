\chapter{Introduction}
\label{chap:intro}

Le projet a pour but de reproduire le jeu des amazones, célèbre pour sa richesse stratégique. C’est un jeu de plateau où le but de chaque joueur est de réduire les déplacements de l'adversaire tout en optimisant les siens et en apportant des modifications au terrain à chaque tour. 

\section{Règles}
\label{sec:regles}

\begin{itemize}
    \item[\textdagger] Déplacements : chaque pion, appelé reine, possède la même couverture de mouvements que la reine aux échecs.
    \medbreak
    \item [\textdagger] Tir : à la fin de chaque déplacement, la reine en question doit tirer une flèche dans une des directions accessibles par cette dernière.
    \medbreak
    \item [\textdagger] Condition de victoire : le jeu se termine lorsque toutes les reines d’un des deux joueurs ne sont plus en capacité de bouger.
    \medbreak
    \item [\textdagger] Client : le jeu doit être réalisé de telle sorte que notre stratégie puisse être autonome afin de jouer une partie contre un autre adversaire sur son propre serveur.
    \medbreak
    \item [\textdagger] Serveur : le jeu doit également pouvoir faire jouer d'autres clients en provenance d'autres groupes de projet. Pour cela, certaines conventions ainsi que des vérifications anti-triche doivent être mises en place.
\end{itemize}

\section{Stratégie de résolution du sujet}
\label{sec:strat_resol}

Ce projet est divisé en deux facettes, le développement des \textit{clients} et le développement du \textit{serveur}. Il est indispensable que les clients soient autonomes et implémentent l'interface définie dans le sujet du projet afin d'être compatible avec l'ensemble des serveurs utilisant celle-ci. Le travail a donc été réparti en conséquence entre client et serveur, pour obtenir un jeu qui fonctionne aussi bien sur un serveur local que sur un serveur externe.\\
Afin d'avoir un client constamment opérationnel, l'utilisation de la méthode de programmation par les tests a été mise en place pour son développement. Les tests ont donc été développés en amont pour permettre au client de progresser petit à petit en résolvant les tests imposés.

\section{Architecture du projet}
\label{sec:architecture}

Le projet se base sur la méthode d'implémentation et de représentation de graphe ainsi que de l'interaction client-serveur. On remarque donc sur le graphe de dépendance en figure \ref{fig:my_label} que les fichiers posant les bases de l'environnement, sont à la base du projet. Ensuite vient le client, qui ne dépend que de peu de fichiers, car il se doit d'être malléable afin de s'adapter lors de son exportation sur des mondes différents et lors de l'exportation vers un autre serveur. \\
En haut de ce graphe, on retrouve le serveur qui est celui qui lie réellement tous les fichiers ensemble et permet le fonctionnement du jeu. On y retrouve notamment la boucle de jeu ainsi que la création du monde et bien entendu, les différents joueurs. C'est lui qui utilise toutes les fonctions de mouvements et de validations des coups. \\
Quant au fichier \texttt{tools}, celui-ci est davantage utile dans la partie test, car il permet l'affichage des matrices GSL facilement.

\medbreak

\begin{figure}[H]
    \centering
    \includegraphics[scale=0.5]{dependency.png}
    \caption{Graph de dépendance du projet}
    \label{fig:my_label}
\end{figure}

\medbreak

Pour garantir le bon fonctionnement des implémentations tout au long du projet, des tests ont été mis en œuvre. Ces derniers sont divisés en plusieurs fichiers, chacun étant destiné à tester un fichier précis du projet. Ce point sera abordé plus détails dans la partie \ref{sec:tests}.